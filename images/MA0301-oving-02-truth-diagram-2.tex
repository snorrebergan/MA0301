\documentclass[tikz]{standalone}

\usetikzlibrary{matrix}

\begin{document}

\begin{tikzpicture}
  \matrix (m) [matrix of math nodes,
  row sep=3em,
  column sep=4em,
  minimum width=2em,
  nodes in empty cells, 
  nodes={anchor=center}]
  {
      &      & p &        &  p & r & t &  \\
    T &    p &   &  q     &    &   &   &  T \\
      &    p & q & \neg r &    & t &   &   \\
  };
  \draw (m-2-1.east) -- ++(2em,0) |- (m-1-3) -- (m-1-4.east) -- ++(1em,0) |- (m-2-4);
  \draw (m-2-1.east) -- ++(2em,0) |- (m-3-2) -- (m-3-3) -- (m-3-4) --%
        (m-3-4.east) -- ++(1em,0) |- (m-2-4);
  \draw (m-2-1.east) -- (m-2-2) -- (m-2-4) -- (m-2-4.east) -- ++(3em,0) |- (m-1-5);

  \draw (m-1-5) -- (m-1-6) -- (m-1-7) -- (m-1-7.east) -- ++(2em,0) |- (m-2-8);
  \draw (m-2-4.east) -- ++(3em,0) |- (m-3-5.center) -- (m-3-6) -- (m-3-7.east) --%
        ++(2em,0) |- (m-2-8);
  % \draw (m-3-4.west) -- ++(-2em,0) |- (m-2-3);
  % \draw ([xshift=2em]m-3-1.east) |- (m-4-2) -- (m-4-3);
  % \draw ([xshift=-2em]m-3-4.west) |- (m-4-3);
  % \draw (m-3-2.east) -- ++(2.5em,0) |- (m-5-2);
  % \draw (m-5-2.west) -- ++(-2.5em,0) |- (m-3-2);
  % \draw (m-1-3.east) -- ++(2.5em,0) |- (m-3-3);
  % \draw (m-3-3.west) -- ++(-2.5em,0) |- (m-1-3);
\end{tikzpicture}

\end{document}